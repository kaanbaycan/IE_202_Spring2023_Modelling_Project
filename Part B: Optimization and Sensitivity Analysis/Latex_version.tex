\documentclass[11pt]{article}
\def\thesection{\arabic{section}.}
\def\thesubsection{\thesection\arabic{subsection}.}
\topmargin -.5in
\textheight 9in
\oddsidemargin -.25in
\evensidemargin -.25in
\textwidth 7in
\parskip .05in
\parindent 0in
\setcounter{secnumdepth}{0}

\renewcommand \baselinestretch{1.1} 
\scriptsize
\include{psfig}
\normalsize

\usepackage{graphicx}
\usepackage{amsmath}
\usepackage{geometry} 
\usepackage{nccmath}
\usepackage{cases}
\usepackage{float}
\usepackage{amsmath,amssymb,amsthm}
\usepackage{mathtools}
\DeclarePairedDelimiter{\ceil}{\lceil}{\rceil}
\DeclarePairedDelimiter\floor{\lfloor}{\rfloor}
\DeclareMathOperator{\E}{\mathbb{E}}
\newcommand{\R}{\mathbb{R}}
\newcommand{\N}{\mathbb{N}}
\newcommand{\Z}{\mathbb{Z}}
\usepackage{framed,multirow}  

\usepackage{graphicx}
\usepackage{amsmath}
\usepackage{cases}
\usepackage{tabularx}
\usepackage[T1]{fontenc}
\usepackage[utf8]{inputenc}
\usepackage{mathtools}
\usepackage{blkarray, bigstrut}
\usepackage{gauss}
\usepackage{xcolor}
\usepackage{epsfig}
\usepackage[framemethod=TikZ]{mdframed}
\mdfdefinestyle{MyFrame}{%
	linecolor=blue,
	outerlinewidth=2pt,
	roundcorner=20pt,
	innertopmargin=\baselineskip,
	innerbottommargin=\baselineskip,
	innerrightmargin=20pt,
	innerleftmargin=20pt,
	backgroundcolor=gray!50!white}

\newcommand{\ba}{\begin{array}}
	\newcommand{\ea}{\end{array}}
\newcommand{\bea}{\begin{eqnarray}}
\newcommand{\eea}{\end{eqnarray}}

\begin{document}
\title{IE 202 INTRODUCTION TO MODELING AND OPTIMIZATION\\ TERM PROJECT\\STAGE 2 \\}
\author{Group No: 22\\Yakup Kaan Baycan 22103383\\ Burak Ünal  22101836 \\ Dilara Demir  22003461}
\maketitle

\newpage


	\newcommand{\Pf}{\mathbf{P}}
	\newcommand{\UP}[2]{\makebox[0pt]{\smash{\raisebox{1.5em}{$\phantom{#2}#1$}}}#2}
	\newcommand{\LF}[1]{\makebox[0pt]{$#1$\hspace{4.5em}}}
	
\begin{center}
{\bf PART A }
\end{center}
In Part A, we solved the mathematical model given to us while making proper assignments between the elements of the company by maximizing the monthly profit. Our objective function is to maximize profit, so we wrote an objective function by collecting the money from all projects, subtracting the penalty cost of undone projects from this all earned money, and then subtracting the salaries in the usual working hour for employees who are assigned to project for different positions  and overworking hour given to the employees for different difficulty levels of projects. By subtracting the total expenses from the total income, we make our objective function in line with the maximize monthly profit that we want. Therefore,  in mathematical model which is given to us, we have objective function and constraints and according to these to solve this problem, we used different solvers to figure out this problem which are Xpress, Cplex, and Gurobi. By using these solvers, we reached an outcome from solvers appropriate lists from our decision variables and maximum monthly profit. In our decision variables we have a project is assigned to a department, a employee is assigned to a project, a project is assigned to some department, a employee is assigned to some project, and total number of overworking hours of employee in a month for a project. With reference to these decision variables from solvers which we used, we examined our objective value is 905000. Here we also had chance to observe which decision variable take which value when optimized.

\begin{center}
{\bf PART B }
\end{center}
In Part B, we revised our model such that our objective function is maximize the number of completed projects and make an observation about how our monthly profit change in this case. At first, we revised our objective function in this report and then make comparison for our monthly profit in Part A and Part B. In new objective function, we try to maximize number of projects are assigned to some department. Again, same as with Part A, we used different solvers which were Xpress, Cplex, and Gurobi to obtain a outcome for this problem. By using these solvers, we reached an outcome from solvers appropriate lists from our decision variables and maximum number of completed projects.In our decision variables we have a project is assigned to a department, an employee is assigned to a project, a project is assigned to some department, an employee is assigned to some project, and total number of overworking hours of employee in a month for a project. According to these solvers, we obtained objective value is found 9  and monthly profit decreased to for Xpress is found 442500, for Cplex is found 506500, and for Gurobi is found 514500. Hence we observe that monthly profit is decreased due to new objective function. In addition, we understand that we have alternate optimal solutions for part b.

\newpage

\begin{center}
{\bf Comparison between Part A and Part B }
\end{center}
While we compare Part A and Part B, we observed that for Xpress, Cplex, and Gurobi monthly profit is decreased as we mentioned in Part B. Then, we focus on why finding 3 different results in 3 different solvers is since even though the number of completed of projects is same which is 9, one of the reasons we found different monthly profits is that the projects to which the employee is assigned differ in the solvers. We observed this from the matrices that we got from all three solvers.Here we observe differences :if we look at [2,1] in all three matrices we encounter:1,0,0 for  $e_1$,$e_2$ and $e_3$ respectively. Hence first solver made a different project assignment to employee 1 than others.

\[
\begin{minipage}{0.5\textwidth}
\[
e_1 = 
\begin{bmatrix}
0 & 1 & 0 & 1 & 0 & 1 & 0 & 1 & 0 & 1 \\
0 & 0 & 0 & 0 & 0 & 1 & 0 & 1 & 0 & 1 \\
0 & 0 & 1 & 1 & 0 & 1 & 0 & 0 & 1 & 1 \\
0 & 0 & 0 & 0 & 0 & 1 & 0 & 0 & 0 & 1 \\
1 & 0 & 1 & 1 & 0 & 1 & 0 & 0 & 0 & 1 \\
0 & 0 & 0 & 1 & 0 & 1 & 0 & 1 & 0 & 1 \\
0 & 0 & 0 & 0 & 0 & 1 & 0 & 0 & 0 & 1 \\
1 & 0 & 1 & 1 & 0 & 0 & 0 & 1 & 0 & 0 \\
0 & 0 & 0 & 0 & 0 & 1 & 0 & 1 & 0 & 1 \\
0 & 1 & 1 & 0 & 0 & 1 & 0 & 1 & 1 & 0 \\
0 & 1 & 0 & 1 & 0 & 0 & 0 & 0 & 1 & 1 \\
0 & 0 & 1 & 1 & 0 & 1 & 0 & 1 & 1 & 0 \\
0 & 0 & 0 & 1 & 1 & 1 & 0 & 1 & 1 & 0 \\
0 & 1 & 1 & 1 & 0 & 0 & 0 & 1 & 1 & 0 \\
0 & 1 & 0 & 1 & 0 & 1 & 0 & 0 & 0 & 1 \\
1 & 0 & 0 & 0 & 1 & 0 & 1 & 1 & 0 & 0 \\
0 & 1 & 0 & 1 & 1 & 1 & 0 & 1 & 0 & 0 \\
0 & 1 & 0 & 1 & 0 & 0 & 1 & 1 & 0 & 1 \\
0 & 1 & 0 & 1 & 0 & 0 & 0 & 0 & 0 & 0 \\
0 & 0 & 0 & 1 & 0 & 1 & 1 & 0 & 0 & 1 \\
\end{bmatrix}
\]
\end{minipage}
\begin{minipage}{0.5\textwidth}
\[
e_2 = 
\begin{bmatrix}
0 & 0 & 1 & 1 & 0 & 1 & 0 & 1 & 1 & 0 \\
0 & 1 & 0 & 0 & 1 & 0 & 0 & 1 & 0 & 0 \\
0 & 0 & 1 & 1 & 1 & 1 & 0 & 0 & 1 & 0 \\
0 & 0 & 0 & 1 & 0 & 0 & 0 & 0 & 0 & 1 \\
0 & 1 & 1 & 1 & 0 & 0 & 0 & 1 & 0 & 1 \\
0 & 1 & 0 & 1 & 0 & 0 & 0 & 1 & 1 & 1 \\
0 & 1 & 0 & 1 & 0 & 1 & 0 & 1 & 0 & 1 \\
0 & 0 & 0 & 1 & 0 & 1 & 0 & 1 & 1 & 1 \\
0 & 0 & 0 & 1 & 0 & 1 & 0 & 1 & 0 & 1 \\
0 & 0 & 0 & 0 & 0 & 0 & 0 & 1 & 0 & 1 \\
0 & 0 & 0 & 0 & 0 & 0 & 0 & 1 & 0 & 1 \\
0 & 1 & 0 & 1 & 0 & 1 & 0 & 1 & 0 & 1 \\
1 & 0 & 0 & 1 & 0 & 1 & 0 & 1 & 1 & 0 \\
0 & 0 & 0 & 1 & 0 & 1 & 0 & 0 & 0 & 0 \\
1 & 0 & 1 & 1 & 0 & 1 & 0 & 0 & 1 & 0 \\
0 & 0 & 0 & 1 & 0 & 0 & 0 & 0 & 0 & 1 \\
0 & 1 & 0 & 1 & 0 & 1 & 0 & 1 & 0 & 1 \\
0 & 1 & 1 & 0 & 0 & 1 & 0 & 1 & 1 & 0 \\
0 & 0 & 1 & 0 & 1 & 1 & 1 & 0 & 0 & 1 \\
1 & 1 & 0 & 0 & 0 & 1 & 1 & 0 & 1 & 0 \\
\end{bmatrix}
\]
\end{minipage}

\[
e_3 = 
\begin{bmatrix}
0 & 0 & 1 & 1 & 0 & 1 & 0 & 1 & 0 & 1 \\
0 & 0 & 0 & 1 & 0 & 0 & 0 & 0 & 0 & 1 \\
0 & 1 & 0 & 1 & 1 & 1 & 0 & 0 & 0 & 1 \\
0 & 1 & 0 & 0 & 0 & 1 & 0 & 1 & 0 & 1 \\
0 & 0 & 0 & 1 & 1 & 1 & 0 & 1 & 0 & 1 \\
0 & 1 & 0 & 1 & 0 & 0 & 0 & 1 & 1 & 1 \\
0 & 1 & 0 & 1 & 0 & 1 & 0 & 1 & 0 & 1 \\
1 & 0 & 0 & 1 & 0 & 1 & 0 & 0 & 0 & 0 \\
1 & 0 & 0 & 1 & 0 & 0 & 0 & 1 & 1 & 1 \\
0 & 0 & 0 & 0 & 0 & 1 & 0 & 0 & 0 & 0 \\
1 & 1 & 0 & 0 & 0 & 0 & 0 & 1 & 1 & 0 \\
0 & 1 & 0 & 1 & 0 & 1 & 0 & 1 & 1 & 0 \\
0 & 0 & 1 & 0 & 0 & 1 & 1 & 0 & 0 & 0 \\
0 & 0 & 1 & 1 & 0 & 0 & 0 & 0 & 1 & 0 \\
0 & 0 & 1 & 1 & 0 & 1 & 0 & 1 & 0 & 1 \\
0 & 0 & 0 & 0 & 1 & 1 & 0 & 0 & 0 & 0 \\
0 & 0 & 1 & 1 & 0 & 1 & 0 & 1 & 1 & 0 \\
0 & 1 & 0 & 1 & 0 & 1 & 0 & 1 & 0 & 1 \\
0 & 0 & 0 & 1 & 0 & 0 & 1 & 0 & 0 & 1 \\
0 & 1 & 0 & 0 & 0 & 1 & 1 & 1 & 0 & 1 \\
\end{bmatrix}
\]



\newpage   
Another reason is that the decision variable, which is the total number of overworking hours of employee in a month for a project, also varies in the solvers. As we can see below from the overworked matrices from all three solvers, they differ. For instance if we examine 4th worker, we see that only the second solver assigned 4th worker with an overwork on 4th project.

\begin{center}
\begin{minipage}{0.6\textwidth}
\[
o_1 = 
\begin{bmatrix}
    0 & 0 & 0 & 0 & 0 & 0 & 0 & 0 & 0 & 0 & 0 & 0 & 0 & 0 & 0 & 0 & 0 & 0 & 0 & 0 \\
    0 & 0 & 0 & 0 & 0 & 0 & 0 & 0 & 0 & 0 & 0 & 0 & 0 & 0 & 0 & 0 & 0 & 0 & 0 & 0 \\
    0 & 0 & 0 & 0 & 0 & 0 & 0 & 0 & 0 & 0 & 0 & 0 & 0 & 0 & 0 & 0 & 0 & 0 & 0 & 0 \\
    0 & 0 & 40 & 0 & 40 & 0 & 0 & 0 & 0 & 0 & 40 & 0 & 0 & 0 & 40 & 0 & 0 & 0 & 40 & 0 \\
    0 & 0 & 0 & 0 & 0 & 0 & 0 & 0 & 0 & 0 & 0 & 0 & 0 & 0 & 0 & 0 & 0 & 0 & 0 & 0 \\
    40 & 40 & 0 & 40 & 0 & 0 & 40 & 0 & 0 & 0 & 0 & 40 & 0 & 0 & 0 & 0 & 0 & 0 & 0 & 40 \\
    0 & 0 & 0 & 0 & 0 & 0 & 0 & 0 & 0 & 0 & 0 & 0 & 0 & 0 & 0 & 0 & 0 & 0 & 0 & 0 \\
    0 & 0 & 0 & 0 & 0 & 40 & 0 & 40 & 40 & 40 & 0 & 0 & 40 & 40 & 0 & 40 & 40 & 40 & 0 & 0 \\
    0 & 0 & 0 & 0 & 0 & 0 & 0 & 0 & 0 & 0 & 0 & 0 & 0 & 0 & 0 & 0 & 0 & 0 & 0 & 0 \\
    0 & 0 & 0 & 0 & 0 & 0 & 0 & 0 & 0 & 0 & 0 & 0 & 0 & 0 & 0 & 0 & 0 & 0 & 0 & 0 \\
\end{bmatrix}
\]
\end{minipage}
\begin{minipage}{0.6\textwidth}
\[
o_2 = 
\begin{bmatrix}
0 & 0 & 0 & 0 & 0 & 0 & 0 & 0 & 0 & 0 & 0 & 0 & 0 & 0 & 0 & 0 & 0 & 0 & 0 & 0 \\
0 & 0 & 0 & 0 & 0 & 0 & 0 & 0 & 0 & 0 & 0 & 0 & 0 & 0 & 0 & 0 & 0 & 0 & 0 & 0 \\
0 & 0 & 0 & 0 & 0 & 0 & 0 & 0 & 0 & 0 & 0 & 0 & 0 & 0 & 0 & 0 & 0 & 0 & 0 & 0 \\
0 & 0 & 0 & 40 & 40 & 0 & 30 & 0 & 0 & 0 & 0 & 0 & 0 & 0 & 40 & 40 & 0 & 0 & 0 & 0 \\
0 & 0 & 0 & 0 & 0 & 0 & 0 & 0 & 0 & 0 & 0 & 0 & 0 & 0 & 0 & 0 & 0 & 0 & 0 & 0 \\
0 & 0 & 0 & 0 & 0 & 0 & 40 & 0 & 0 & 0 & 0 & 0 & 40 & 40 & 0 & 0 & 0 & 40 & 40 & 40 \\
0 & 0 & 0 & 0 & 0 & 0 & 0 & 0 & 0 & 0 & 0 & 0 & 0 & 0 & 0 & 0 & 0 & 0 & 0 & 0 \\
40 & 40 & 0 & 0 & 40 & 10 & 0 & 40 & 40 & 40 & 40 & 40 & 0 & 0 & 0 & 0 & 40 & 0 & 0 & 0 \\
0 & 0 & 0 & 0 & 0 & 0 & 0 & 0 & 0 & 0 & 0 & 0 & 0 & 0 & 0 & 0 & 0 & 0 & 0 & 0 \\
0 & 0 & 0 & 0 & 0 & 0 & 0 & 0 & 0 & 0 & 0 & 0 & 0 & 0 & 0 & 0 & 0 & 0 & 0 & 0 \\
\end{bmatrix}
\]
\end{minipage}
\end{center}

\[
o_3
\begin{bmatrix}
0 & 0 & 0 & 0 & 0 & 0 & 0 & 0 & 0 & 0 & 0 & 0 & 0 & 0 & 0 & 0 & 0 & 0 & 0 & 0 \\
0 & 0 & 0 & 0 & 0 & 0 & 0 & 0 & 0 & 0 & 0 & 0 & 0 & 0 & 0 & 0 & 0 & 0 & 0 & 0 \\
0 & 0 & 0 & 0 & 0 & 0 & 0 & 0 & 0 & 0 & 0 & 0 & 0 & 0 & 0 & 0 & 0 & 0 & 0 & 0 \\
40 & 0 & 0 & 0 & 30 & 0 & 0 & 0 & 0 & 0 & 0 & 0 & 0 & 40 & 0 & 0 & 40 & 0 & 40 & 0 \\
0 & 0 & 0 & 0 & 0 & 0 & 0 & 0 & 0 & 0 & 0 & 0 & 0 & 0 & 0 & 0 & 0 & 0 & 0 & 0 \\
0 & 0 & 40 & 0 & 0 & 0 & 0 & 40 & 0 & 40 & 0 & 40 & 40 & 0 & 0 & 40 & 0 & 0 & 0 & 40 \\
0 & 0 & 0 & 0 & 0 & 0 & 0 & 0 & 0 & 0 & 0 & 0 & 0 & 0 & 0 & 0 & 0 & 0 & 0 & 0 \\
0 & 0 & 0 & 0 & 0 & 0 & 0 & 0 & 0 & 0 & 40 & 0 & 0 & 0 & 0 & 0 & 0 & 0 & 0 & 0 \\
0 & 0 & 0 & 0 & 0 & 0 & 0 & 0 & 0 & 0 & 0 & 0 & 0 & 0 & 0 & 0 & 0 & 0 & 0 & 0 \\
0 & 40 & 0 & 40 & 10 & 40 & 40 & 0 & 40 & 0 & 0 & 0 & 0 & 0 & 40 & 0 & 0 & 40 & 0 & 0 \\
\end{bmatrix}
\]  
 In addition to this, we also noticed that in our solvers, they can't solve project 8 in all three of them. Therefore, when we can conclude that when we compare objective functions of Part A and Part B (for Part A maximize monthly profit and for Part B maximize the number of completed projects) we examine that 9 projects are assigned. However, monthly profit is decreased in Part B in all solvers.
    \end{enumerate}
{\bf } 
 \begin{enumerate} 
  
    \end{enumerate} 
 
 {\bf } 
 \begin{enumerate} 
 
    
    \end{enumerate}    
    \begin{mdframed}[style=MyFrame]
{\textbf{Parameters:}}  


\begin{fleqn}





\end{mdframed}

\begin{mdframed}


        


        
\end{fleqn}


\end{mdframed}
 


{\textbf{}} 



\end{mdframed}

\maketitle
\begin{center}
    

\section{Part C}
\end{center}
At part C, there is a completely new problem. New problem is stated below:
\item Yigit&Bora Partners is a consulting firm that wants to assign its four senior members; Yigit, Bora,
Gozde, and Michelle to four companies, namely; Amazor, ProctorGambler, Pegasos, Sielens. Each
company has 3 available projects. It means there are 12 available projects in total. The hourly
works must be done to complete each project is given below.
\\

{\bf Assumptions:} 
 \begin{enumerate} 
   \item An employee can do all the projects.
   \item More than 1 employee could assign to a single project of any companies.
   \item The given hourly works of companies and specific projects is equal to the required minimum total working hours of employees.
   \end{enumerate}
\\
{\bf English Description of Constraints:} 
\begin{enumerate}
   \item Bora can't work more than 80 hours.
   \item Michelle can't work more than 80 hours.
   \item Any employee can not work less than 60 hours.
   \item Any employee can not work more than 1000 hours.
   \item Gözde needs to work more than 150 hours on project 1.
   \item Gözde can not work at project 3.
   \item Michelle does not work at project 1 of any company.
   \item Yiğit needs to work at least 30 hours with company Proctor&Gambler.
   \item Number of hours worked on a project must be bigger than required working hours for each project. (Stated in parameters)
   \item Total working hour for any employee is bigger than 0. (This constraint is redundant for now but it can help us to avoid problems for newly introduced specıal cases. 
   
\end{enumerate}

{\bf Why this choice of variables:} 
 \begin{enumerate} 
   \item $X_{ijk}$ variables are used to determine how many hours i-th employee works on the j-th company's k-th project.

\end{enumerate}

 \begin{mdframed}[style=MyFrame]

{\textbf{Parameters:}}  
\begin{fleqn}
\\
$c_i =$ I-th employee's cost per working hour. $i=1,2,\dots,4$

$h_{ij} =$ Required number of hours for i-th company's j-th project. \\ $i=1,2,\dots,4$, $j=1,2,3$
\end{fleqn}
\
\end{mdframed}

 \begin{mdframed}[style=MyFrame]

{\textbf{Decision Variables:}}  
\begin{fleqn}
\\
$X_{ijk}$ = i-th employee works on the j-th company's k-th project.  $i=1,2,\dots,4$,  $j=1,2,\dots,4$,  $i=1,2,3$
\end{fleqn}
\
\end{mdframed}

{\begin{mdframed}[style=MyFrame]
{\textbf{Model:}
\\\\
\max\left\sum_{i=1}^{4}\sum_{j=1}^{4}\sum_{k=1}^{3} c_iX_i_j_k\right \\
Subject to\\
1.\quad \sum_{j}^{4}\sum_{k=1}^{3} X_{2jk}\leq 80\right \\
2.\quad \sum_{j}^{4}\sum_{k=1}^{3} X_{4jk}\leq 80\right \\
3.\quad \sum_{j}^{4}\sum_{k=1}^{3} X_{ijk}\geq 60\right\quad i=1,2,\dots,4 \\
4.\quad \sum_{j}^{4}\sum_{k=1}^{3} X_{ijk}\leq 1000 \quad i=1,2,\dots,4\right \\
5.\quad \sum_{k=1}^{3} X_{31k}\leq 150\right \\
6.\quad \sum_{k}^{3} X_3_3_k = 0 \right \\
7.\quad \sum_{k}^{3} X_4_1_k = 0 \right \\
8.\quad \sum_{j}^{4} X_1_j_2\geq 30 \right \\
9.\quad \sum_{k}^{3} X_i_j_k\geq H_j_k\quad k=1,2,\dots,4 \quad j=1,2,3 \right \\
10.\quad X_i_j_k\geq 0 \quad i=1,2,\dots,4 \quad j=1,2,3 \quad k=1,2,\dots,4


{\begin{align}
\end{align}
\end{mdframed}


{\begin{mdframed}[style=MyFrame]
{\textbf{Data:}
\\\\

{\bf ${h_j_k}$ $Data$ $Matrix$}

\[
\begin{bmatrix}
13 & 70 & 45 & 60 \\
45 & 66 & 33 & 80 \\
60 & 55 & 98 & 40 \\
\end{bmatrix}
\]

{\bf ${c_i}$ $Data$ $Matrix$}

\[
\begin{bmatrix}
100 \\
85  \\
90  \\
80  \\
\end{bmatrix}
\]

{\begin{align}
\end{align}
\end{mdframed}

\newpage


{\bf Sensitivity Analysis Analysis:}\\\\
First Table: Decision Variables\\
From the decision variables table we can see how many hours a person worked at each project from the "Final Value" column. Reduced cost column shows us the coefficients of row zero on optimal table for any variable (for our example it is X which states how many hours a person worked at a company's any job) this means it indicates how much our objective function value will change for unit change in that variable. "Objective value" column is hourly pay for our example and we can see how we can change hourly pay without changing our objective function value from the "Allowable increase" and "Allowable decrease" columns. To be more specific, for example for the M3 row from the first table we can see Yiğit's hourly payment is 100 dollars from objective coefficient value column and by looking allowable increase and allowable decrease columns we can find the range [100,110] for objective coefficient value of M3 row. Similarly, here are intervals for objective coefficient value ranges for non zero final value cells:\\
\\ M3 - [100,110]\hspace{3cm}L5 - [0,100]
\\ R3 - [90, 100]\hspace{3.1cm}N5 - [0,100]
\\ J4 - [ 85, 85]\hspace{3.2cm}O5 - [0,100]
\\ P4 - [ 85, 85]\hspace{3.15cm}Q5 - [0,100]
\\ S4 - [-15, 85]\hspace{3.2cm}R5 - [0,100]
\\ H5 - [ 0, 100]\hspace{3.2cm}J6 - [80,80]
\\ I5 - [ 0, 100]\hspace{3.25cm}M6 - [70,80]
\\ K5 - [ 0, 100]
\\ \\For zero final value cells interval logic is the same. However we also see infinity values there, reason for that is if someone is not working at a certain project, it is not surprising that that person will not work if their salary per hour goes to infinity since this is a cost minimization problem but if we decrease their salary more than the given interval they MAY work on that specific project which can change the solution.
\\ From the reduced cost column we can see how much change is needed for a non basic variable to become a basic variable. For example for the constraint "Yiğit Amazor1" that value is 10.
\\ \\Second Table: Constraints\\
By looking the Final Value columns we can see the value of our constraints at the optimal solution we found. For example Gözde Project1 constraint's final value is 208, this shows us our constraint about Gözde's total working hour on any project 1 is 208. Furhtermore, from the allowable increase or from allowable decrease columns we can see what are the limits if we want to change the right hand side of that specific constraint. Returning to our Gözde Project 1 example we can see since on our solution that constraint's value is 208 we can only increase the current Right hand side value (150) at most 58. After that point our current solution will not satisfy the new requirements. We need the check the optimality of our solution. Since we are changing the RHS we may disturb the primal optimality and as a result of that we may need to use dual simplex to return to optimality. Even though we are in the ranges of the allowable increase and decrease thus our solution is still optimal, our optimal solution value will change according to our constraint's shadow price. To visualize it, "Hours Worked Pegasos1" constraint has 90 shadow price, this means if we increase this contraint by 5 units our optimal solution value (which is not wanted since this is a minimization pronblem) will also increase 5 times our constraint's shadow price which is 450. For futher analysis if RHS value and Final values are equal to each other we can say that it is a binding constraint. To give an example, "Hours Worked P\&G2" constraint is a binding constraint.
\\ \\Also there is another report that is called "Feasibility Report" from this report we can see our total cost which is 59580 and this is our optimal solution value. Any action that makes this value to increase is an unwanted outcome since this problem is a minimization problem and we want our total cost low as possible. 
\begin{table}[htbp]
  \centering
  \caption{Feasibility Report}
    \begin{tabular}{llrr}
    \toprule
    \multicolumn{1}{c}{\textcolor[rgb]{ 0,  0,  .502}{\textbf{Cell}}} & \multicolumn{1}{c}{\textcolor[rgb]{ 0,  0,  .502}{\textbf{Name}}} & \multicolumn{1}{c}{\textcolor[rgb]{ 0,  0,  .502}{\textbf{First Value}}} & \multicolumn{1}{c}{\textcolor[rgb]{ 0,  0,  .502}{\textbf{Last Value}}} \\
    \midrule
    \$H\$14 & Total Cost  & 59580 & 59580 \\
    \bottomrule
    \end{tabular}%
  \label{tab:addlabel}%
\end{table}%




\begin{table}[htbp]
  \centering
  \small
  \caption{Decision Variables}
    \begin{tabular}{|c|c|p{1cm}|p{2cm}|p{2cm}|p{2cm}|p{2cm}|}
    \toprule
    \textcolor[rgb]{ .267,  .329,  .416}{\textbf{Cells}} & \textcolor[rgb]{ .267,  .329,  .416}{\textbf{Name}} & \textcolor[rgb]{ .267,  .329,  .416}{\textbf{Final Value}} & \textcolor[rgb]{ .267,  .329,  .416}{\textbf{Reduced Costs}} & \textcolor[rgb]{ .267,  .329,  .416}{\textbf{Objective Value}} & \textcolor[rgb]{ .267,  .329,  .416}{\textbf{Allowable Increase}} & \textcolor[rgb]{ .267,  .329,  .416}{\textbf{Allowable Decrease}} \\
    \midrule
    H3    & Yiğit Amazor1 & 0     & 10    & 100   & 1E+100 & 10 \\
    I3    & Yiğit Amazor2 & 0     & 10    & 100   & 1E+100 & 10 \\
    J3    & Yiğit Amazor3 & 0     & 0     & 100   & 1E+100 & 0 \\
    K3    & Yiğit P\&G 1 & 0     & 10    & 100   & 1E+100 & 10 \\
    L3    & Yiğit P\&G 2 & 0     & 10    & 100   & 1E+100 & 10 \\
    M3    & Yiğit P\&G 3 & 30    & 0     & 100   & 10,0000001 & 0 \\
    N3    & Yiğit Pegasos1 & 0     & 10    & 100   & 1E+100 & 10 \\
    O3    & Yiğit Pegasos 2 & 0     & 10    & 100   & 1E+100 & 10 \\
    P3    & Yiğit Pegasos 3 & 63    & 0     & 100   & 0     & 10,0000001 \\
    Q3    & Yiğit Sielens 1 & 0     & 10    & 100   & 1E+100 & 10 \\
    R3    & Yiğit Sielens 2 & 0     & 10    & 100   & 1E+100 & 10 \\
    S3    & Yiğit Sielens 3 & 0     & 0     & 100   & 1E+100 & 0 \\
    H4    & Bora Amazor1 & 0     & 10    & 85    & 1E+100 & 10 \\
    I4    & Bora Amazor2 & 0     & 10    & 85    & 1E+100 & 10 \\
    J4    & Bora Amazor3 & 5     & 0     & 85    & 0     & 0 \\
    K4    & Bora P\&G 1 & 0     & 10    & 85    & 1E+100 & 10 \\
    L4    & Bora P\&G 2 & 0     & 10    & 85    & 1E+100 & 10 \\
    M4    & Bora P\&G 3 & 0     & 0     & 85    & 1E+100 & 0 \\
    N4    & Bora Pegasos1 & 0     & 10    & 85    & 1E+100 & 10 \\
    O4    & Bora Pegasos 2 & 0     & 10    & 85    & 1E+100 & 10 \\
    P4    & Bora Pegasos 3 & 35    & 0     & 85    & 0     & 0 \\
    Q4    & Bora Sielens 1 & 0     & 10    & 85    & 1E+100 & 10 \\
    R4    & Bora Sielens 2 & 0     & 10    & 85    & 1E+100 & 10 \\
    S4    & Bora Sielens 3 & 40    & 0     & 85    & 0     & 100,0000001 \\
    H5    & Gözde Amazor1 & 13    & 0     & 90    & 10,0000001 & 90,0000001 \\
    I5    & Gözde Amazor2 & 45    & 0     & 90    & 10,0000001 & 90,0000001 \\
    J5    & Gözde Amazor3 & 0     & 0     & 90    & 1E+100 & 1E+100 \\
    K5    & Gözde P\&G 1 & 70    & 0     & 90    & 10,0000001 & 90,0000001 \\
    L5    & Gözde P\&G 2 & 66    & 0     & 90    & 10,0000001 & 90,0000001 \\
    M5    & Gözde P\&G 3 & 0     & 0     & 90    & 1E+100 & 1E+100 \\
    N5    & Gözde Pegasos1 & 45    & 0     & 90    & 10,0000001 & 90,0000001 \\
    O5    & Gözde Pegasos 2 & 33    & 0     & 90    & 10,0000001 & 90,0000001 \\
    P5    & Gözde Pegasos 3 & 0     & 0     & 90    & 1E+100 & 1E+100 \\
    Q5    & Gözde Sielens 1 & 80    & 0     & 90    & 10,0000001 & 90,0000001 \\
    R5    & Gözde Sielens 2 & 60    & 0     & 90    & 10,0000001 & 90,0000001 \\
    S5    & Gözde Sielens 3 & 0     & 0     & 90    & 1E+100 & 1E+100 \\
    H6    & Michelle Amazor1 & 0     & 10    & 80    & 1E+100 & 10 \\
    I6    & Michelle Amazor2 & 0     & 10    & 80    & 1E+100 & 10 \\
    J6    & Michelle Amazor3 & 55    & 0     & 80    & 0     & 0 \\
    K6    & Michelle P\&G 1 & 0     & 10    & 80    & 1E+100 & 10 \\
    L6    & Michelle P\&G 2 & 0     & 10    & 80    & 1E+100 & 10 \\
    M6    & Michelle P\&G 3 & 25    & 0     & 80    & 0     & 10,0000001 \\
    N6    & Michelle Pegasos1 & 0     & 10    & 80    & 1E+100 & 10 \\
    O6    & Michelle Pegasos 2 & 0     & 10    & 80    & 1E+100 & 10 \\
    P6    & Michelle Pegasos 3 & 0     & 0     & 80    & 1E+100 & 0 \\
    Q6    & Michelle Sielens 1 & 0     & 10    & 80    & 1E+100 & 10 \\
    R6    & Michelle Sielens 2 & 0     & 10    & 80    & 1E+100 & 10 \\
    S6    & Michelle Sielens 3 & 0     & 0     & 80    & 1E+100 & 0 \\
    \bottomrule
    \end{tabular}%
\label{tab:addlabel}%
\end{table}%

% Table generated by Excel2LaTeX from sheet 'Sayfa1 Sensitivity'
\begin{table}[htbp]
  \centering
  \small
  \caption{Constraints}
    \begin{tabular}{|c|c|p{1cm}|p{1.5cm}|p{1.5cm}|p{1.5cm}|p{1.5cm}|}
    \toprule
    \textcolor[rgb]{ .267,  .329,  .416}{\textbf{Cells}} & \textcolor[rgb]{ .267,  .329,  .416}{\textbf{Name}} & \textcolor[rgb]{ .267,  .329,  .416}{\textbf{Final Value}} & \textcolor[rgb]{ .267,  .329,  .416}{\textbf{Shadow Price}} & \textcolor[rgb]{ .267,  .329,  .416}{\textbf{RHS Value}} & \textcolor[rgb]{ .267,  .329,  .416}{\textbf{Allowable Increase}} & \textcolor[rgb]{ .267,  .329,  .416}{\textbf{Allowable Decrease}} \\
    \midrule
    B19>=150 & Gözde Project1  & 208   & 0     & 150   & 58    & 1E+100 \\
    B20>=30 & Yiğit P\&G  & 30    & 0     & 30    & 5     & 30 \\
    H6=0  & Michelle Amazor1 & 0     & 0     & 0     & 0     & 0 \\
    H7>=H8 & Hours Worked Amazor1 & 13    & 90    & 13    & 588   & 13 \\
    I7>=I8 & Hours Worked Amazor2 & 45    & 90    & 45    & 588   & 45 \\
    J5=0  & Gözde Amazor3 & 0     & -10   & 0     & 5     & 0 \\
    J7>=J8 & Hours Worked Amazor3 & 60    & 100   & 60    & 35    & 5 \\
    K6=0  & Michelle P\&G 1 & 0     & 0     & 0     & 0     & 0 \\
    K7>=K8 & Hours Worked P\&G 1 & 70    & 90    & 70    & 588   & 58 \\
    L7>=L8 & Hours Worked P\&G 2 & 66    & 90    & 66    & 588   & 66 \\
    M5=0  & Gözde P\&G 3 & 0     & -10   & 0     & 5     & 0 \\
    M7>=M8 & Hours Worked P\&G 3 & 55    & 100   & 55    & 35    & 5 \\
    N6=0  & Michelle Pegasos1 & 0     & 0     & 0     & 0     & 0 \\
    N7>=N8 & Hours Worked Pegasos1 & 45    & 90    & 45    & 588   & 45 \\
    O7>=O8 & Hours Worked Pegasos 2 & 33    & 90    & 33    & 588   & 33 \\
    P5=0  & Gözde Pegasos 3 & 0     & -10   & 0     & 33    & 0 \\
    P7>=P8 & Hours Worked Pegasos 3 & 98    & 100   & 98    & 907   & 33 \\
    Q6=0  & Michelle Sielens 1 & 0     & 0     & 0     & 0     & 0 \\
    Q7>=Q8 & Hours Worked Sielens 1 & 80    & 90    & 80    & 588   & 58 \\
    R7>=R8 & Hours Worked Sielens 2 & 60    & 90    & 60    & 588   & 60 \\
    S5=0  & Gözde Sielens 3 & 0     & -10   & 0     & 33    & 0 \\
    S7>=S8 & Hours Worked Sielens 3 & 40    & 100   & 40    & 35    & 33 \\
    T3<=1000 & Yiğit Total Working Hour & 93    & 0     & 1000  & 1E+100 & 907 \\
    T4<=1000 & Bora Total Working Hour & 80    & 0     & 1000  & 1E+100 & 920 \\
    T5<=1000 & Gözde Total Working Hour & 412   & 0     & 1000  & 1E+100 & 588 \\
    T6<=1000 & Michelle Total Working Hour & 80    & 0     & 1000  & 1E+100 & 920 \\
    T3>=60 & Yiğit Total Working Hour & 93    & 0     & 60    & 33    & 1E+100 \\
    T4>=60 & Bora Total Working Hour & 80    & 0     & 60    & 20    & 1E+100 \\
    T5>=60 & Gözde Total Working Hour & 412   & 0     & 60    & 352   & 1E+100 \\
    T6>=60 & Michelle Total Working Hour & 80    & 0     & 60    & 20    & 1E+100 \\
    T4<=80 & Bora Total Working Hour & 80    & -15   & 80    & 33    & 20 \\
    T6<=80 & Michelle Total Working Hour & 80    & -20   & 80    & 5     & 20 \\
    \bottomrule
    \end{tabular}%
  \label{tab:addlabel}%
\end{table}%






\end{document}



	
	\end{document}
\
